\documentclass[12pt]{article}

\usepackage{sbc-template}
\usepackage{graphicx,url}
\usepackage[brazil]{babel} 
\usepackage[utf8]{inputenc}
\usepackage{comment}

     
\sloppy

\title{Planejamento da construção de aeroportos na cidade de Belém utilizando o método de clusterização \textit{k-means}}

\author{Eduardo Gil S. Cardoso\inst{1}, Gabriela S. Maximino\inst{1}, Igor Matheus S. Moreira\inst{1} }

\address{
  Instituto de Ciências Exatas e Naturais -- Faculdade de Computação\\
  Universidade Federal do Pará -- Belém, PA -- Brasil
\email{\{gabriela.maximino,igor.moreira\}@icen.ufpa.br}\email{\{eduardo.gil.s.cardoso\}@gmail.com}
}

\begin{document} 

\maketitle

\begin{abstract}
  This article demonstrates the application of k-means algorithm to the clustering problem involving the construction of three airports in the city of Belém, Pará. This report is part of the deliverable associated to the task proposed by professor Reginaldo Cordeiro dos Santos Filho for the Artificial Intelligence course, taught under the Computer Science Bachelor's degree program at the Federal University of Pará.
\end{abstract}
     
\begin{resumo} 
  Este artigo demonstra a aplicação do algoritmo k-means para o problema de clusterização envolvendo a construção de três aeroportos na cidade de Belém, Pará. Este trabalho é parte do entregável relativo à tarefa proposta pelo Prof. Dr. Reginaldo Cordeiro dos Santos Filho para a disciplina de Inteligência Artificial, ministrada sob o curso de Bacharelado em Ciência da Computação na Universidade Federal do Pará.
\end{resumo}


\section{Introdução}
- falar do problema dado;
- falar do algoritmo k-means e porque escolhemos ele;



\section{Descrição da base de dados}
A base de dados construída é composta pelas coordenadas geográficas do ponto central de cada bairro da cidade de Belém. Como resultado, obteve-se um banco de dados contendo 71 instâncias e 2 \textit{features}, que são 'Bairro' e 'Coordenadas'. As coordenadas estão no formato coord-x, coord-y.



[foto da base?]

\section{Metodologia do trabalho}
O trabalho foi dividido em três etapas: construção da base de dados; implementação do algoritmo e realização de testes; escrita do artigo.

A construção da base de dados ficou a cargo de um estudante. Para essa etapa, inicialmente verificou-se a lista de bairros da cidade de Belém no site da prefeitura¹. Em seguida, buscou-se a coordenada de cada bairro utilizando a ferramenta Nominatin do projeto \textit{OpenStreetMap}, a qual fornece a delimitação e o ponto central dos bairros. Ao todo, 71 pontos centrais foram coletados e armazenados em uma planilha no formato .csv.

A implementação do algoritmo [...]

Para a escrita do artigo, dois membros do grupo foram necessários, enquanto o terceiro ficou responsável pela revisão geral do trabalho. 



% ¹fonte da prefeiturahttp://www.belem.pa.gov.br/segep/download/mapas/bairros/bairros_index.htm
% fonte do nominatim
\section{Resultados}

\section{Conclusão}


\bibliographystyle{sbc}
\bibliography{sbc-template}

\end{document}
